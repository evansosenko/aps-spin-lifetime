\appendix*
\section{Computation}

In this appendix we derive an expression for
the nonlocal resistance for finite contact resistance.
We first present the key definitions and critical boundary conditions.
We then derive the relation between the nonlocal resistance
and the spin chemical potential at the far contact, $μ_s^N(L)$.
Finally, we solve the diffusion equation inside the semiconductor to find $μ_s^N(L)$.

\subsection{Definitions}

Many of the definitions and results in this section are taken from
\cite{ActaPhysicaSlovaca.57.4_5.565-907}.
The chemical potential and spin chemical potential are defined in terms
of the spin up and spin down chemical potentials,
\begin{subequations}
  \label{eq:potentials}
  \begin{alignat}{2}
    & μ   && = \frac{1}{2} \left( μ_↑ + μ_↓ \right) , \\
    & μ_s && = \frac{1}{2} \left( μ_↑ - μ_↓ \right) .
  \end{alignat}
\end{subequations}
The material conductances and polarization are defined in terms
of the spin up and spin down conductances,
\begin{subequations}
  \label{eq:conductances}
  \begin{alignat}{2}
    & σ   && = σ_↑ + σ_↓ , \\
    & σ_s && = σ_↑ - σ_↓ , \\
    & P_σ && = \frac{σ_s}{σ} .
  \end{alignat}
\end{subequations}
The gradient of the chemical potentials drives a current and spin current,
\begin{subequations}
  \label{eq:currents}
  \begin{alignat}{3}
    & J_{↑↓} && = σ_{↑↓} ∇μ_{↑↓} , \\
    %
    \label{eq:currents.current}
    & J      && = J_↑ + J_↓      & = σ   ∇μ + σ_s ∇μ_s , \\
    %
    \label{eq:currents.spincurrent}
    & J_s    && = J_↑ - J_↓      & = σ_s ∇μ + σ   ∇μ_s .
  \end{alignat}
\end{subequations}
To indicate the material, any of the above can have a superscript
$N$ (normal semiconductor) or $F$ (ferromagnet).

The contact conductances and polarization are defined in terms
of the spin up and spin down contact conductances,
\begin{subequations}
  \label{eq:contact_conductances}
  \begin{alignat}{2}
    & Σ   && = Σ_↑ + Σ_↓ , \\
    & Σ_s && = Σ_↑ - Σ_↓ , \\
    & P_Σ && = \frac{Σ_s}{Σ} .
  \end{alignat}
\end{subequations}
The mismatch of the chemical potentials across the contact
drives a current and spin current,
\begin{subequations}
  \label{eq:contact_currents}
  \begin{alignat}{2}
    & J_{↑↓}^C && = Σ_{↑↓} \left( μ^N_{↑↓} - μ^F_{↑↓} \right)_c , \\
    %
    & J^C      && = J_↑^C + J_↓^C , \\
    & J_s^C    && = J_↑^C - J_↓^C .
  \end{alignat}
\end{subequations}
The subscript $c$ will always denote the function evaluated at the contact.

We will use the term current to refer to $J$,
when in fact this is a particle current density.
For constant $J$, the physical charge current $I$ will be related to $J$
by a relation $I = - A J / e$ for some characteristic area $A$.

To reduce the number of subscripts and superscripts in the following,
we adopt the notation for the potentials
\begin{subequations}
  \begin{equation}
    \begin{aligned}
    & \begin{alignedat}{2}
        & u && = μ^N_s , \\
        & v && = μ^N   ,
      \end{alignedat}
    & \begin{alignedat}{2}
        & φ && = μ^F_s , \\
        & ψ && = μ^F   ,
      \end{alignedat}
    \end{aligned}
  \end{equation}
  and currents
  \begin{equation}
    \begin{alignedat}{2}
      & ȷ   && = J_s     , \\
      & J_c && = J^C     , \\
      & ȷ_c && = J_s^C   .
    \end{alignedat}
  \end{equation}
\end{subequations}

We rewrite \cref{eq:contact_currents} as
\begin{subequations}
  \label{eq:contact_currents.2}
  \begin{alignat}{3}
    \label{eq:contact_currents.2.current}
    & J_c && = Σ   \left( v_c - ψ_c \right) && + Σ_s \left( u_c - φ_c \right) , \\
    %
    \label{eq:contact_currents.2.spincurrent}
    & ȷ_c && = Σ_s \left( v_c - ψ_c \right) && + Σ   \left( u_c - φ_c \right) ,
  \end{alignat}
\end{subequations}
and \cref{eq:conductances,eq:currents} as
\begin{equation}
  \label{eq:bdry_current}
  ȷ = P_σ J + 4 \frac{σ_↑ σ_↓}{σ} ∇μ_s .
\end{equation}

Using \cref{eq:contact_conductances,eq:contact_currents.2},
\begin{equation}
  \label{eq:bdry_current_contact}
  ȷ_c = P_Σ^i J_c + {R_C^i}^{-1} \left( u_c - φ_c \right) ,
\end{equation}
where the contact resistance is
\begin{equation}
  R_C^i = \frac{Σ^i}{4 Σ_↑^i Σ_↓^i} .
\end{equation}
The superscript $i$ allows for contacts with difference conductances.

\subsection{Boundary conditions}

\subsubsection{Semiconductor}

For the semiconductor, $σ^N_↑ = σ^N_↓ = σ^N / 2$, so $P_σ^N = 0$.
Evaluating \cref{eq:bdry_current} at the contact gives
\begin{equation}
  \label{eq:bdry_current.semiconductor}
  ȷ^N_c = σ^N ( ∇u )_c .
\end{equation}

\subsubsection{Ferromagnet}

For the ferromagnet, one assumes $μ_s^F$ satisfies
the one dimensional diffusion equation.
We choose the $z'$ coordinate antiparallel to $z$ with origin at the contact.
The equation
\begin{equation}
  \label{eq:diffusion.ferromagnet}
  φ'' \left( z' \right) - k_F^2 φ \left( z' \right) = 0 ,
\end{equation}
with the boundary condition
$\lim_{z' → - ∞} φ(z') = 0$
has solution
\begin{equation}
  \label{eq:diffusion.ferromagnet.solution}
  φ(z') = φ_c e^{k_F z'} ,
\end{equation}
where $φ_c = φ(0)$ is a yet undetermined constant.
Putting this into \cref{eq:bdry_current} and evaluating it at the contact gives
\begin{equation}
  \label{eq:bdry_current.ferromagnet}
  ȷ^F_c = P_σ^F J^F_c + R_F^{-1} φ_c ,
\end{equation}
where the ferromagnet resistance is
\begin{equation}
  R_F = \frac{σ^F}{ 4 σ_↑^F σ_↓^F k_F } .
\end{equation}
Here, $λ_F = 1 / k_F$ is the spin diffusion length in the ferromagnet.

\subsubsection{Continuity assumptions}

At the contact, the current and spin current are assumed continuous,
\begin{subequations}
  \label{eq:continuity.current}
  \begin{alignat}{3}
    & J_c && = J^F_c && = J^N_c , \\
    & ȷ_c && = ȷ^F_c && = ȷ^N_c .
  \end{alignat}
\end{subequations}
Using \cref{eq:bdry_current.ferromagnet,eq:bdry_current_contact,eq:continuity.current}
we find the relation
\begin{subequations}
  \begin{equation}
    \label{eq:bdry_solutions.current}
    \left( P_σ^F R_F + P_Σ^i R_C^i \right) J_c = \left( R_F + R_C^i \right) ȷ_c - u_c ,
  \end{equation}
  and that $φ_c$ is determined by
  \begin{equation}
    \label{eq:bdry_solutions.potential}
    R_F^{-1} φ_c = \frac{ \left( P_Σ^i - P_σ^F \right) R_C^i ȷ_c + P_σ^F u_c}{P_σ^F R_F + P_Σ^i R_C^i} .
  \end{equation}
\end{subequations}
In the special case of zero current at the contact ($J_c = 0$),
\cref{eq:bdry_solutions.current,eq:bdry_solutions.potential} reduce to
\begin{subequations}
  \label{eq:bdry_solutions.zero}
  \begin{alignat}{2}
    \label{eq:bdry_solutions.zero.current}
    & ȷ_c && = \frac{1}{R_F + R_C^i} u_c   , \\
    %
    \label{eq:bdry_solutions.zero.potential}
    & φ_c && = \frac{R_F}{R_F + R_C^i} u_c .
  \end{alignat}
\end{subequations}

\subsection{Nonlocal resistance}

In this section we derive the precise relation between $\rNL$ and $μ_s^N(L)$.
Note that we may write in general, for some $\bar{μ}$,
\begin{equation}
  μ = \bar{μ} + P_σ μ_s ,
\end{equation}
and, following
\cite{PhysRevB.67.052409},
define the voltage due to the difference in the chemical potentials across the contacts by
\begin{equation}
  V_c = \left( \bar{μ}_c^N - \bar{μ}_c^F \right) / e.
\end{equation}

We assume a fixed current $J_0 = \abs{J_0} > 0$
flows down through the contact at $x = 0$
and to the left in the semiconductor for $x ≤ 0$,
and no current flows for $x > 0$.
The experimentally measured quantity is the
ohmic nonlocal resistance $\rNL^Ω = V_L / I_0$,
where $I_0 = - W L J_0 / e$ is the current through the contact at $x = 0$.
It is convenient to introduce the effective nonlocal resistance $\rNL$
defined by
\begin{equation}
  \rNL = W L \rNL^Ω = - e V_L / J_0 = \frac{\bar{μ}_c^F - \bar{μ}_c^N}{J_0} .
\end{equation}
To determine $\rNL$, we must express the difference
of these chemical potentials in terms of $μ_s^N(L)$.

Since there are two ferromagnetic contacts,
we have separate functions $ψ$ and $φ$ for each contact
which we will denote by $ψ^0$, $φ^0$, and $ψ^L$, $φ^L$.
From \cref{eq:diffusion.ferromagnet.solution}, we have
\begin{subequations}
  \begin{align}
    φ^0 \left( z' \right) & = φ_0 e^{k_F z'} , \\
    φ^L \left( z' \right) & = φ_L e^{k_F z'} .
  \end{align}
\end{subequations}

The physical restriction on the current flow in the semiconductor
is imposed by noting that since $σ_s^N = 0$,
\cref{eq:currents.current} gives $J^N = σ^N ∇v$, so we must have
\begin{equation}
  v_x (x) =
    \begin{cases}
      v_x (0) - \left( J_0 / σ^N \right) x & \text{ for } x ≤ 0 , \\
      v_x (0)                              & \text{ for } x > 0 ,
    \end{cases}
\end{equation}
$v_y (x) = v_y (0)$, and $v_z (x) = v_z (0)$.

Using \cref{eq:currents.current},
the restriction on the current flow in each ferromagnet gives
\begin{subequations}
  \begin{align}
    ∇ψ^0 & = \left( J_0 / σ^F \right) - P_σ^F ∇φ^0 , \\
    ∇ψ^L & = - P_σ^F ∇φ^L  ,
  \end{align}
\end{subequations}
Integrating and enforcing
$e V_c = v_x (0) - \left( ψ_c - P_σ^F φ_c \right)$,
\begin{subequations}
  \begin{align}
    ψ^0 \left( z' \right) & = - e V_0 + P_σ^F φ_0 \left( 2 - e^{k_F z'} \right) + \left( J_0 / σ^F \right) z' + v_x (0) , \\
    ψ^L \left( z' \right) & = - e V_L + P_σ^F φ_L \left( 2 - e^{k_F z'} \right) + v_x (0) .
  \end{align}
\end{subequations}

There is no current at the contact at $x = 0$, thus \cref{eq:contact_currents.2.current} gives
\begin{equation}
  ψ_L - v_L = P_Σ^L \left( u_L - φ_L \right) .
\end{equation}
and with
\cref{eq:bdry_solutions.zero.potential},
we find
\begin{equation}
  \begin{aligned}
    \rNL & = \left( ψ_L - v_L \right) - P_σ^F φ_L \\
         & = \left[ P_Σ^L \left( 1 - \frac{R_F}{R_F + R_C^L} \right) - \frac{P_σ^F R_F}{R_F + R_C^L} \right] \frac{u_x (L)}{J_0} .
  \end{aligned}
\end{equation}

\subsection{Diffusion equation}

In this section we show how to solve for $μ_s^N(L)$.
Inside the semiconductor, $u$ satisfies the diffusion equation
\begin{equation}
  \label{eq:diffusion}
  D ∇^2 u - \frac{u}{τ} + ω × u = 0 ,
\end{equation}
Here, $D$ is the diffusion constant, $τ$ the spin lifetime,
and $ω = \left( g μ_B / ℏ \right) B$ is proportional to the applied magnetic field
(with $g$ the gyromagnetic ratio and $μ_B$ the Bohr magneton).
The spin diffusion length is $λ = \sqrt{D τ}$.
