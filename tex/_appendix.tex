\appendix*
\section{Computation}

\subsection{Definitions}

Most of the following is taken from
\cite{ActaPhysicaSlovaca.57.4_5.565-907}.
The chemical potential and spin chemical potential are defined in terms
of the spin up and spin down chemical potentials,
\begin{subequations}
  \label{eq:potentials}
  \begin{alignat}{2}
    & μ   && = \frac{1}{2} \left( μ_↑ + μ_↓ \right) , \\
    & μ_s && = \frac{1}{2} \left( μ_↑ - μ_↓ \right) .
  \end{alignat}
\end{subequations}
The material conductances and polarization are defined in terms
of the spin up and spin down conductances,
\begin{subequations}
  \label{eq:conductances}
  \begin{alignat}{2}
    & σ   && = σ_↑ + σ_↓ , \\
    & σ_s && = σ_↑ - σ_↓ , \\
    & P_σ && = \frac{σ_s}{σ} .
  \end{alignat}
\end{subequations}
The gradient of the chemical potentials drives a current and spin current,
\begin{subequations}
  \label{eq:currents}
  \begin{alignat}{3}
    & J_{↑↓} && = σ_{↑↓} ∇μ_{↑↓} , \\
    %
    \label{eq:currents.current}
    & J      && = J_↑ + J_↓      & = σ   ∇μ + σ_s ∇μ_s , \\
    %
    \label{eq:currents.spincurrent}
    & J_s    && = J_↑ - J_↓      & = σ_s ∇μ + σ   ∇μ_s .
  \end{alignat}
\end{subequations}
To indicate the material, any of the above can have a superscript
$N$ (normal semiconductor) or $F$ (ferromagnet).

The contact conductances and polarization are defined in terms
of the spin up and spin down contact conductances,
\begin{subequations}
  \label{eq:contact_conductances}
  \begin{alignat}{2}
    & Σ   && = Σ_↑ + Σ_↓ , \\
    & Σ_s && = Σ_↑ - Σ_↓ , \\
    & P_Σ && = \frac{Σ_s}{Σ} .
  \end{alignat}
\end{subequations}
The mismatch of the chemical potentials across the contact
drives a current and spin current,
\begin{subequations}
  \label{eq:contact_currents}
  \begin{alignat}{2}
    & J_{↑↓}^C && = Σ_{↑↓} \left( μ^N_{↑↓} - μ^F_{↑↓} \right)_c , \\
    %
    & J^C      && = J_↑^C + J_↓^C , \\
    & J_s^C    && = J_↑^C - J_↓^C .
  \end{alignat}
\end{subequations}
The subscript $c$ will always denote the function evaluated at the contact.

To reduce the number of subscripts and superscripts in the following,
we adopt the notation for the potentials
\begin{subequations}
  \label{eq:potentials.alt}
  \begin{equation}
    \begin{aligned}
    & \begin{alignedat}{2}
        & u && = μ^N_s , \\
        & v && = μ^N   ,
      \end{alignedat}
    & \begin{alignedat}{2}
        & φ && = μ^F_s , \\
        & ψ && = μ^F   ,
      \end{alignedat}
    \end{aligned}
  \end{equation}
\end{subequations}
and currents
\begin{subequations}
  \label{eq:currents.alt}
  \begin{alignat}{2}
    & ȷ   && = J_s     , \\
    & J_c && = J^C     , \\
    & ȷ_c && = J_s^C   .
  \end{alignat}
\end{subequations}
