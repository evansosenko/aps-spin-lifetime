\section{Model}
\label{s:model}

\begin{figure}
  \caption{
    The geometry of the nonlocal spin valve analyzed in this paper is shown.
    There are two ferromagnetic electrodes placed on a conducting channel.
    Current $I$ flows into the left electrode,
    while the potential $V$ is measured at the right electrode.
    The nonlocal resistance is defined as the ratio $V / I$.
    For spin dependent phenomena, the relevant quantity of interest
    is the difference between the nonlocal resistance for the parallel
    and antiparallel orientations of magnetization of the two electrodes.
   }
  \label{fig:nonlocal_spin_valve}
  \input{components/tikz-nonlocal_spin_valve/tex/_head}
  \begin{tikzpicture}[scale=0.7]
    \input{components/tikz-nonlocal_spin_valve/tex/_nonlocal_spin_valve}
  \end{tikzpicture}
\end{figure}

The assumed device geometry is shown in \cref{fig:nonlocal_spin_valve},
and the nonlocal resistance is $\rNL = V / I$ \cite{PhysRevB.67.052409}
(see also \cref{s:appendix:nonlocal_resistance}).
The key parameters are
the contact spacing $L$,
the diffusion constant $D$,
the spin lifetime $τ$,
the spin diffusion length $λ = \sqrt{D τ}$,
and $ω = \left( g μ_B / ℏ \right) B$ which is proportional to the applied magnetic field $B$.
Any symbols not defined in the body of this paper can be found in \cref{s:appendix}.

We find the nonlocal resistance may be written in the form
\begin{equation}
  \label{eq:nonlocal_resistance}
  \rNL^± = ± p_1 p_2 R_N f ,
\end{equation}
where $p_1, p_2$ depend on the polarizations, $R_N$ is a resistance scale,
and $f$ is unitless and depends only on the scales $L / λ$, $ω τ$, and $λ / r_i$.
The overall sign corresponds to parallel and antiparallel ferromagnetic alignments.
The parameters $r_i$ with $i$ either $0$ for the left contact or $L$ for the right are
\begin{equation}
  \label{eq:r-parameter}
  r_i = \frac{R_F + R_C^i}{\rSQ} W ,
\end{equation}
where $R_F$ is the resistances of the ferromagnet
and $R_C^i$ are the contact resistances
\footnote{
  The resistances $R_F, R_C^i$ are the effective resistances
  of a unit cross sectional area.
  To obtain an expression in terms of the ohmic resistances,
  one must make the substitutions
  $R_F → W_F W R_F$,
  $R_C^i → W_F W R_C^i$,
  where $W_F$ is the contact width, i.e., $W_F W$ is the contact area.
  We will use the same symbols for either resistance type when the meaning is clear.
},
$W$ is the graphene flake width, and
\begin{equation}
  \label{eq:square_resistance}
  \rSQ = W / σ^N ,
\end{equation}
is the graphene square (sheet) resistance.

A detailed derivation is given in \cref{s:appendix}.
Specifically, we find a resistance
\begin{equation}
  R_N = \frac{λ}{W L} \frac{1}{σ^N} ,
\end{equation}
polarizations given by
\begin{subequations}
  \label{eq:polarizations}
  \begin{align}
    p_1 & = \frac{P_σ^F R_F + P_Σ^L R_C^L}{R_F + R_C^L} , \\
    p_2 / p_1 & = \left. \left( 1 - \frac{P_σ^F R_F}{P_Σ^L R_C^L} \right) \middle/
                  \left(1 + \frac{P_σ^F R_F}{P_Σ^L R_C^L} \right) \right. ,
  \end{align}
\end{subequations}
and the function
\begin{widetext}
  \begin{equation}
    \label{eq:f}
    f = \re{ \left\{ \left(
          2 \left[ \sqrt{1 + i ω τ} + \frac{λ}{2} \left( \frac{1}{r_0} + \frac{1}{r_L} \right) \right]
          e^{\left( L / λ \right) \sqrt{1 + i ω τ}}
          + \frac{λ^2}{r_0 r_L} \frac{
              \sinh{ \left[ \left( L / λ \right) \sqrt{1 + i ω τ} \right] }}
            {\sqrt{1 + i ω τ}}
        \right)^{-1} \right\} } .
  \end{equation}
\end{widetext}

The expression $Δ \rNL = \abs{\rNL^+ - \rNL^-}$
measures the difference in signal between parallel and antiparallel field alignments.
We combine $P^2 = \abs{p_1 p_2}$
\footnote{
  Assuming the polarizations $P_σ^F$ and $P_Σ^L$ have the same sign bounds $P ≤ 1$.
},
and write
\begin{equation}
  \label{eq:nonlocal_resistance.difference}
  Δ \rNL = 2 P^2 R_N \abs{f} ,
\end{equation}
with
\begin{equation}
  R_N = \frac{λ}{W} \frac{1}{σ_G} ,
\end{equation}
where $σ_G = σ^N L$ is the graphene conductance normally given in units of
$\si{\milli \siemens} = \left( \si{\milli \ohm} \right)^{-1}$.
