\section{Model}
\label{s:model}

The assumed device geometry and definition of the nonlocal resistance
is the same as given in figure 1 from \cite{PhysRevB.67.052409}.
The key parameters are
the contact spacing $L$,
the diffusion constant $D$,
the spin lifetime $τ$,
the spin diffusion length $λ = \sqrt{D τ}$,
and $ω = \left( g μ_B / ℏ \right) B$ which is proportional to the applied magnetic field $B$.
Any symbols not defined in the body of this paper can be found in \cref{s:appendix}.

We find the nonlocal resistance may be written in the form
\begin{equation}
  \rNL^± = ± p_1 p_2 R_N f ,
\end{equation}
where $p_1, p_2$ depend on the polarizations, $R_N$ is a resistance scale,
and $f$ is unitless and depends only on the scales $L / λ$, $ω τ$, and $λ / r_i$.
The sign corresponds to parallel and antiparallel field alignments.
The parameters $r_i$ are proportional to the contact resistances $R_C^i$ as
\begin{equation}
  \label{eq:r-parameter}
  r_i = \frac{R_F + R_C^i}{\rSQ} W ,
\end{equation}
where $R_F$ is the resistances of the ferromagnet, $W$ is graphene flake width, and
\begin{equation}
  \label{eq:square_resistance}
  \rSQ = W / σ^N .
\end{equation}
\footnote{
  The resistances $R_F, R_C^i$ are the effective resistances
  of a unit cross sectional area.
  To obtain an expression in terms of the ohmic resistances,
  one must make the substitutions
  $R_F → W_F W R_C^i$,
  $R_C^i → W_F W R_C^i$,
  where $W_F$ is the contact width, i.e., $W_F W$ is the contact area.
}.
A detailed derivation is given in \cref{s:appendix},
specifically, we find a resistance
\begin{equation}
  R_N = \frac{λ}{W L} \frac{1}{σ^N} ,
\end{equation}
polarizations
\begin{subequations}
  \label{eq:polarizations}
  \begin{align}
    p_1 & = \frac{P_σ^F R_F + P_Σ^L R_C^L}{R_F + R_C^L} , \\
    p_2 & = p_1^2 \left. \left( 1 - \frac{P_σ^F R_F}{P_Σ^L R_C^L} \right) \middle/ \left(1 + \frac{P_σ^F R_F}{P_Σ^L R_C^L} \right) \right. ,
  \end{align}
\end{subequations}
and the function
\begin{widetext}
  \begin{equation}
    \label{eq:f}
    f = \re{ \left\{ \left( 2 \left[ \sqrt{1 + i ω τ} + \frac{λ}{2} \left( \frac{1}{r_0} + \frac{1}{r_L} \right) \right] e^{\left( L / λ \right) \sqrt{1 + i ω τ}} + \frac{λ^2}{r_0 r_L} \frac{\sinh{\left( L / λ \right) \sqrt{1 + i ω τ}}}{\sqrt{1 + i ω τ}} \right)^{-1} \right\} } .
  \end{equation}
\end{widetext}

The expression $Δ \rNL = \abs{\rNL^+ - \rNL^-}$
measures the difference in signal between parallel and antiparallel field alignments.
Assuming positive polarizations allows us to combine
$P^2 = \abs{p_1 p_2} < 1$, and write
\begin{equation}
  Δ \rNL = 2 P^2 R_N \abs{f} ,
\end{equation}
with
\begin{equation}
  R_N = \frac{λ}{W} \frac{1}{σ_G} ,
\end{equation}
where $σ_G = σ^N L$ is the graphene conductance normally given in units of
$\si{\milli \siemens} = \left( \si{\milli \ohm} \right)^{-1}$.
