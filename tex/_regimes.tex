\section{Regimes}
\label{s:regimes}

In this section we discuss various limits of our expression.
We show agreement of our expression with past results
and discuses how fitting Hanle

In the limit of tunneling contacts, $R_C^0, R_C^L ≫ R_F$.
Putting $r_0, r_L → ∞$ gives $p_1 p_2 → \left( P_Σ^L \right)^2$ and
\begin{equation}
  f = \re{\frac{e^{- \left( L / λ \right) \sqrt{1 + i ω τ}}}{2 \sqrt{1 + i ω τ}}} ,
\end{equation}
which is of the same form as found in appendix B of
\cite{PhysRevB.37.5312}.
In the limit of zero magnetic field,
\begin{equation}
  Δ \rNL = \left( P_Σ^L \right)^2 R_N e^{- L/λ} ,
\end{equation}
which agrees with eq. (6) in
\cite{PhysRevB.67.052409}.

In the following, we only consider the case $r = r_0 = r_L$ of similar contacts.
Let $f_0$ denote $f$ at zero magnetic field,
\begin{equation}
  f_0 = \left[ 2 \left( 1 + λ / r \right) e^{L / λ} + \left( λ / r \right)^2 \sinh{L / λ} \right]^{-1} ,
\end{equation}
which agrees with eq. (3) in
\cite{PhysRevB.80.214427}.
For $λ / r ≫ 1$,
\begin{equation}
  f_0 = \frac{2 e^{- \left( L / λ \right)}}{\left( λ / r \right)^2} .
\end{equation}

When $\sqrt{ω τ} ≫ λ / r ≫ 1$, we approximate $1 + i ω τ ≈ i ω τ$,
\begin{equation}
  f = \frac{1}{2 \sqrt{ω τ}}
      e^{- \left( L / λ \right) \sqrt{ω τ / 2}}
      \cos{\left[ \frac{L}{λ} \sqrt{\frac{ω τ}{2}} + \frac{π}{4} \right]} .
\end{equation}
In this limit, the nonlocal resistance scales as
\begin{equation}
  Δ \rNL ∝ \frac{λ}{\sqrt{ω τ}} = \sqrt{\frac{D}{ω}} ,
\end{equation}
and the normalized nonlocal resistance as
\begin{equation}
  f / f_0 ∝ \frac{\left( λ / r \right)^2}{\sqrt{ω τ}} .
\end{equation}

When $λ / r ≫ \sqrt{ω τ} ≫ 1$,
\begin{equation}
  f = \frac{\sqrt{ω τ}}{\left( λ / r \right)^2}
      e^{- \left( L / λ \right) \sqrt{ω τ / 2}}
      \sin{\left[ \frac{L}{λ} \sqrt{\frac{ω τ}{2}} + \frac{π}{4} \right]} .
\end{equation}
In this limit, the nonlocal resistance scales as
\begin{equation}
  Δ \rNL ∝ λ \sqrt{ω τ} = τ \sqrt{ω D} ,
\end{equation}
and the normalized nonlocal resistance as
\begin{equation}
  f / f_0 ∝ \sqrt{ω τ} .
\end{equation}

\subsection{Effective $g$-factor}

In \cite{Swartz2013}, it is shown that when using the standard integral formula
\begin{equation}
  \rNL = \sNL ∫_0^∞ \frac{e^{-t / τ}}{\sqrt{4 π D t}}
             \exp{\left[- \frac{L^2}{4 π D t} \right]} \cos{ω t} \: dt
\end{equation}
Hanle fits are non-unique up to the scaling
$g → c g$, $τ → τ / c$, $D → c D$, $sNL → c$
for any constant $c > 0$.
