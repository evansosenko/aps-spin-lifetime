\section{Regimes}
\label{s:regimes}
In this section we discuss the various limits of
the expression describing the Hanle precession curve.
First we show that the commonly used results for zero magnetic field
and tunneling contacts is correctly reproduced.
Next, we discuss regimes where appropriate scaling will give non-unique Hanle fits.
In the following, we mostly consider the case $r = r_0 = r_L$ of similar contacts.

In the limit of tunneling contacts, $R_C^0, R_C^L ≫ R_F$.
Putting $r_0, r_L → ∞$ gives $p_1 p_2 → \left( P_Σ^L \right)^2$ and
\begin{equation}
  f = \re{\frac{e^{- \left( L / λ \right) \sqrt{1 + i ω τ}}}{2 \sqrt{1 + i ω τ}}} ,
\end{equation}
which is of the same form as found in appendix B of
\cite{PhysRevB.37.5312}.
Fitting with this expression was found to give results equivalent to fitting with
\cref{eq:nonlocal_resistance.integral}.
In the limit of zero magnetic field,
\begin{equation}
  Δ \rNL = \left( P_Σ^L \right)^2 R_N e^{- L / λ} ,
\end{equation}
which agrees with eq. (6) in
\cite{PhysRevB.67.052409}.

Let $f_0$ denote $f$ at zero magnetic field,
\begin{equation}
  f_0 = \left[ 2 \left( 1 + λ / r \right) e^{L / λ} + \left( λ / r \right)^2 \sinh{L / λ} \right]^{-1} ,
\end{equation}
which agrees with eq. (3) in
\cite{PhysRevB.80.214427}.

\subsection{Scaling}

For $λ / r ≫ 1$,
\begin{equation}
  f_0 = \frac{2 e^{- \left( L / λ \right)}}{\left( λ / r \right)^2} .
\end{equation}
When $\sqrt{ω τ} ≫ 1$, we approximate $1 + i ω τ ≈ i ω τ$.
Note that the product
\begin{equation}
   \frac{L}{λ} \sqrt{ω τ} = L \sqrt{\frac{D}{ω}}
\end{equation}
is independent of the lifetime.

When $\sqrt{ω τ} ≫ λ / r ≫ 1$,
\begin{equation}
  \label{eq:regime.1.f}
  f = \frac{1}{2 \sqrt{ω τ}}
      e^{- \left( L / λ \right) \sqrt{ω τ / 2}}
      \cos{\left[ \frac{L}{λ} \sqrt{\frac{ω τ}{2}} + \frac{π}{4} \right]} .
\end{equation}
In this limit, the nonlocal resistance scales as
\begin{equation}
  \label{eq:regime.1.nonlocal_resistance}
  Δ \rNL ∝ \frac{λ}{\sqrt{ω τ}} = \sqrt{\frac{D}{ω}} ,
\end{equation}
and the normalized nonlocal resistance as
\begin{equation}
  \label{eq:regime.1.ratio}
  f / f_0 ∝ \frac{\left( λ / r \right)^2}{\sqrt{ω τ}} = D \sqrt{\frac{τ}{r^4 ω}} .
\end{equation}

When $λ / r ≫ \sqrt{ω τ} ≫ 1$,
\begin{equation}
  \label{eq:regime.2.f}
  f = \frac{\sqrt{ω τ}}{\left( λ / r \right)^2}
      e^{- \left( L / λ \right) \sqrt{ω τ / 2}}
      \sin{\left[ \frac{L}{λ} \sqrt{\frac{ω τ}{2}} + \frac{π}{4} \right]} .
\end{equation}
In this limit, the nonlocal resistance scales as
\begin{equation}
  \label{eq:regime.2.nonlocal_resistance}
  Δ \rNL ∝ λ \sqrt{ω τ} = τ \sqrt{ω D} ,
\end{equation}
and the normalized nonlocal resistance as
\begin{equation}
  \label{eq:regime.2.ratio}
  f / f_0 ∝ \sqrt{ω τ} .
\end{equation}

In the limits of \cref{eq:regime.1.f,eq:regime.2.f},
the zeros of the Hanle fit are independent of the lifetime
and are determined by $D$ though the condition
\begin{equation}
  \frac{L}{λ} \sqrt{\frac{ω τ}{2}} + \frac{π}{4} = \frac{n π}{2} ,
\end{equation}
where $n = 1$ for \cref{eq:regime.1.f} and $n = 0$ for \cref{eq:regime.2.f}.
When fitting the normalized nonlocal resistance
in the limit of \cref{eq:regime.1.ratio},
scaling $τ → c τ$ and $r → c^{1/4} r$ by some constant $c > 0$ will give the same fit.
Thus, one can obtain a fit with larger lifetime through moderate increase in $r$.
This scaling was observed in numerical simulations
\cite{PhysRevB.86.235408}.

\subsection{Effective $g$-factor}

In \cite{Swartz2013}, it is shown that when using the standard integral formula
\begin{equation}
  \label{eq:nonlocal_resistance.integral}
  \rNL = \sNL ∫_0^∞ \frac{e^{-t / τ}}{\sqrt{4 π D t}}
             \exp{\left[- \frac{L^2}{4 π D t} \right]} \cos{ω t} \: dt ,
\end{equation}
Hanle fits are non-unique up to the scaling
$g → c g$, $τ → τ / c$, $D → c D$, $\sNL → c$
for any constant $c > 0$.
This non-unique scaling problem equally applies to the model presented here.
This is clear given \cref{eq:nonlocal_resistance}
as $λ$ and $ω τ$ remain unchanged under the scaling
$g → c g$, $τ → τ / c$, $D → c D$.
